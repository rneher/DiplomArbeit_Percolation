%25.09.03

\chapter{Zusammenfassung und Ausblick}
\label{sec:zusammenfassung}


\section*{Euler-Charakteristik und Perkolationsschwellen in zwei Dimensionen}
In der Einleitung wurde die Beobachtung vorgestellt, dass die site-Perkolationsschwelle $p_c$ jedes archimedischen Gitters immer zwischen der Nullstelle $p_0$ und dem Wendepunkt $p_0^{(2)}$ der mittleren Euler-Charakteristik liegt. Genauer gilt f\"ur alle archimedischen Gitter $p_0^{(2)}<p_c<p_0$, und das arithmetische Mittel von $p_0$ und $p_0^{(2)}$ liefert eine exzellente Approximation von $p_c$. Dieser Befund war der Ausgangspunkt f\"ur weitere Untersuchungen in zwei Dimensionen. Dazu wurde die Euler-Charakteristik einer Vielzahl zweidimensionaler Gitter berechnet, und mit den, zumeist numerisch bestimmten, Perkolationsschwellen verglichen. Dar\"uberhinaus liefert das Skalenverhalten der Cluster bei $p_c$ Argumente f\"ur die G\"ultigkeit der Beobachtung.
\begin{itemize}
\item Die Perkolationsschwellen aller ``self-matching'' Gitter fallen mit der Nullstelle der Euler-Charakteristik zusammen. Auch f\"ur die eindimensionale Kette und das Bethe-Gitter gilt $p_0=p_c$.
\item F\"ur site-Perkolation auf Laves-Gittern, auf 2-uniformen Gittern und auf einigen irregul\"aren Gittern gilt $p_0^{(2)}\leq p_c \leq p_0$. Die site-Perkolationsschwellen der 2-uniformen Gitter waren nicht bekannt und wurden numerisch bestimmt. 
\item Die Euler-Charakteristiken der \"Uberdeckungsgitter wurden f\"ur alle archimedischen Gitter und f\"ur das Voronoi-Gitter berechnet. Mit Ausnahme des $(3,4,6,4)$-Gitters gilt f\"ur alle untersuchten Gitter $p_0^{(2)}\leq p_c \leq p_0$ f\"ur $p_c\geq \frac{1}{2}$ und $p_0^{(2)}\geq p_c \geq p_0$ f\"ur $p_c\leq \frac{1}{2}$.
\item Die site-Perkolationsschwellen zuf\"allig dekorierter Mosaike wurden numerisch bestimmt und die Euler-Charakteristiken der dekorierten Mosaike berechnet. Die Nullstelle der Euler-Charakteristik $p_0(m)$ und die Perkolationsschwelle $p_c(m)$ \"andern sich mit dem Bruchteil $m$ der dekorierten Plaketten. F\"ur $p_c(m)>\frac{1}{2}$ ist $p_0(m)>p_c(m)$ und f\"ur $p_c(m)<\frac{1}{2}$ gilt $p_0(m)<p_c(m)$. F\"ur den Wendepunkt der Euler-Charakteristik gilt dieses Verhalten i. A. nicht.
\item Das Skalenverhalten der Cluster in der N\"ahe des kritischen Punktes liefert Argumente daf\"ur, dass $p_c<p_0$ gilt, wenn $p_c>\frac{1}{2}$ ist, und umgekehrt, dass $p_c>p_0$ gilt, wenn $p_c<\frac{1}{2}$ ist. Dazu wurde die mittlere Euler-Charakteristik pro Vertex $\chi_{s_0}(p_c)$ am kritischen Punkt betrachtet, zu der nur Cluster und L\"ocher der Gr\"o"se $s\geq s_0$ beitragen. Entsprechend wurde $[p_c-q_c]_{s_0}$ als die Differenz der Wahrscheinlichkeiten, dass ein Vertex Teil eines Clusters oder eines Loches der Gr\"o"se $s\geq s_0$ ist, definiert. Mit der Skalenannahme erh\"alt man f\"ur $s_0\gg 1$ f\"ur das Verh\"altnis beider Gr\"o"sen
\begin{equation}
  \frac{\chi_{s_0}(p_c)}{[p_c-q_c]_{s_0}}=\frac{5}{96s_0}.
\end{equation}
$\chi_{s_0}(p_c)$ hat also f\"ur $s_0\gg 1$ das gleiche Vorzeichen wie $[p_c-q_c]_{s_0}$.
\item F\"ur Gitter, die eine Substruktur enthalten, gelten die diskutierten Beziehungen i. A. nicht. Sowohl f\"ur bond-Perkolation, als auch f\"ur site-Perkolation konnten Beispiele gefunden werden, in denen exakte Perkolationsschwellen bekannt sind, die \"uber der Nullstelle liegen, obwohl $p_c>\frac{1}{2}$ ist. Bei anderen Gitter liegen die Perkolationsschwellen unter dem Wendepunkt. 
\end{itemize}
Die Beziehungen $p_0^{(2)}\leq p_c \leq p_0$ f\"ur $p_c\geq \frac{1}{2}$ und $p_0^{(2)}\geq p_c \geq p_0$ f\"ur $p_c\leq \frac{1}{2}$ konnten also als ``Faustregel'' f\"ur zweidimensionale Gitter ohne nennenswerte Substruktur empirisch gut best\"atigt werden. Die Skalenargumente unterst\"utzen die Vermutung, dass Abweichungen von der Faustregel mit Substrukturen der Gitter zusammenh\"angen. Durch die Gegenbeispiele sind die Hoffnungen auf einen allgemeinen Beweis der Vermutung aber zunichte gemacht worden. Ob die Ungleichungen unter bestimmten Regularit\"atsannahmen richtig sind, bleibt offen.

\section*{Euler-Charakteristik und Perkolationsschwellen in drei Dimensionen}
F\"ur einige dreidimensionale Gitter war bekannt, dass die Nullstelle der mittleren Euler-Charakteristik \"uber der Perkolationsschwelle liegt. 
In drei Dimensionen f\"uhrt die Abh\"angigkeit der Euler-Charakteristik von der Wahl der Zellen auf Mehrdeutigkeiten. Die Berechnung der Euler-Charakteristik dreidimensionaler Gitter wird immer dann kompliziert, wenn die WSZ des Gitters mehr Fl\"achen hat, als ein Gittervertex Nachbarn hat.  
Es wurden die mittlere Euler-Charakteristik einer Reihe weiterer Gitter berechnet. Auch in drei Dimensionen best\"atigt sich die Vermutung, dass die Nullstelle der Euler-Charakteristik gr\"o"ser als die site-Perkolationsschwelle ist, in den allermeisten F\"allen. 
\section*{Bond-site-Perkolation}
Um die Euler-Charakteristik von bond-site-Perkolationsprozessen bestimmen zu k\"onnen, wurde die im Kapitel \ref{sec:mixed} beschriebene Methode entwickelt. Mit dieser Methode kann die Geometrie von Figuren unter beliebig vorgegebenen Zusammenhangsverh\"altnissen berechnet werden. Die Methode fand au"serdem Anwendung in der Berechnung der Euler-Charakteristik von Gittern, die ansonsten nur mit komplizierten Zellen durchzuf\"uhren ist. Es wurden die Euler-Charakteristiken f\"ur bond-site-Perkolation auf dem zweidimensionalen Dreiecks- und Quadratgitter, sowie auf dem dreidimensionalen sc- und fcc-Gitter berechnet. Abgesehen vom Dreiecksgitter, liegt die Nullstelle in der $(p_s,p_b)$-Ebene jenseits von der kritischen Kurve, analog zu $p_c<p_0$ bei reiner site-Perkolation. Beide Kurven haben in allen F\"allen \"ahnliche Form und liegen in etwa so dicht zusammen, wie die $p_0$ und $p_c$ der reinen bond- und site-Perkolation. Die Faustregel scheint daher auch auf bond-site-Perkolation anwendbar zu sein.


\section*{Ausblick}
Die Ergebnisse dieser Arbeit best\"atigen die Faustregeln:
\begin{itemize}
\item  $p_0^{(2)}\leq p_c \leq p_0$ f\"ur $p_c\geq \frac{1}{2}$ und $p_0^{(2)}\geq p_c \geq p_0$ f\"ur $p_c\leq \frac{1}{2}$ f\"ur Perkolation auf zweidimensionalen Gittern 
\item $p_c<p_0$ f\"ur dreidimensionale Gitter
\item F\"ur bond-site-Perkolation liegt die Kurve mit $\chi(p_s,p_b)=0$ \"uber der kritischen Kurve.
\end{itemize}
Allerdings wurde auch deutlich, dass diese Beziehungen keineswegs immer gelten. Abweichungen von der Regel konnten mit Substruktur der Gitter in Verbindung gebracht werden. Wie sich Substrukturen im Detail auf die Euler-Charakteristik und Perkolation auswirken, und ob es Klassen von Gittern gibt, bei denen die Faustregeln rigoros gelten, bleibt offen.
\\

Die Tatsache, dass der beobachtete Zusammenhang zwischen der Nullstelle der Euler-Charakteristik und der Perkolationsschwelle nicht nur f\"ur ausgew\"ahlte site-Perkolations"-pro"-bleme gilt, sondern auch f\"ur zuf\"allig dekorierte Mosaike und sogar f\"ur bond-site-Per"-ko"-la"-tion gilt, l\"asst hoffen, dass die Faustregel f\"ur eine gro"se Klasse von Perkolationsproblemen anwendbar ist. F\"ur die G\"ultigkeit scheint nicht so sehr die Art des Perkolationsprozesses, als viel mehr die Homogenit\"at der verwendeten Gitter entscheidend zu sein.